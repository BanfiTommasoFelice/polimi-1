\documentclass[12pt, a4paper]{report}

\usepackage[margin=2cm]{geometry}
\usepackage{ragged2e}
\usepackage{amssymb}
\usepackage{amsmath}
\usepackage{hyperref}

\newtheorem{definition}{Definizione}

\title{Cheat Sheet}
\author{Gabr1313}
\date{\today}

\begin{document}
\justify
\sloppy
\maketitle

\section*{Programmazione concorrente}
\begin{itemize}
	\item Conviene guardare un contesto alla volta e spesso nel metre si trovano
		anche i deadlock.
	\item nel caso di una deadlock, nello schema si mettono tutti i valori che
		global puo' assumere da quando sono state eseguite le istruzioni che
		implicheranno tale deadlock.
\end{itemize}

\newpage
\section*{Gestione dei processi}
\begin{itemize}
	\item Il padre prosegue l'esecuzione dopo la creazione del figlio
	\item Il task che da attesa passa a pronto viene messo in esecuzione
	\item Alla exit di un processo, anche i thread figli vengono terminati
	\item Quando un processo termina, anche se non scritta esplicitamente,
		bisogna riportare la \texttt{exit}
\end{itemize}


\newpage
\section*{Moduli del kernel}
\begin{itemize}
	\item Segui lo schema della del cheat sheet
	\item prima di ogni ritorno in modalità U, vai a \texttt{schedule} e
		\texttt{pick\_next\_task}. Al ritorno da \texttt{pick\_next\_task} salvi
		sulla pila l'\texttt{USP} e poi avviene il context switch.
	\item Scrivi sulla pila tutte le chiamate, e tira una riga sopra a quella da
		cui ritorni
	\item segui le frecce finchè bisogna (in base al contesto, non sempre fino
		alla fine)
		\begin{itemize}
			\item \texttt{resched(set TIF\_NEED\_RESCHED)} avviene solo se
				effittivamente è necessario un rescheduling: se vieni da
				\texttt{check\_preempt\_curr} ci devono essere processi in
				stato di pronto, mentre se viene da \texttt{task\_tick}, devi
				assicurarti che il timer sia scaduto per il quanto di tempo.
			\item \texttt{TIF\_NEED\_RESCHED} è la condizione che determina la
				chiamate a \texttt{schedule} nel momento del ritorno in
				modalità U.
			\item \texttt{schedule} setta sempre \texttt{TIF\_NEED\_RESCHED} a
				0.
		\end{itemize}
	\item si usa la notazione \texttt{a *1* da *2*} per indicare che la funzione
		1 ha chiamato la funzione 2, e quindi sullo stack ci si salva
		l'indirizzo a cui ritornare nella funzione 1.
	\item \texttt{R\_int(evento)}: chiamata prima \texttt{task\_tick()} $\to
			\dots$ e poi \texttt{controlla\_timer()} $\to \dots$.\\
		L' \texttt{s\_stack} è del tipo:
		\begin{itemize}
			\item \texttt{USP}
			\item \dots
			\item \texttt{rientro a R\_int(CK) da schedule}
			\item \dots
			\item \texttt{PSR U}
			\item \texttt{rientro a CU da R\_Int(evento)} (codice utente)
		\end{itemize}
		Mentre in un interrupt annidato 
		\begin{itemize}
			\item \dots
			\item \texttt{PSR S}
			\item \texttt{rientro a R\_Int\_1(evento) da R\_Int\_2(evento)}
			\item \dots
		\end{itemize}
	\item \texttt{syscall}: dipende da syscall a syscall.\\
		La chimata a è del tipo:
		\begin{itemize}
			\item \texttt{> *funzione\_libreria*}
			\item \texttt{> ***}
			\item \texttt{> syscall}
			\item \texttt{> System\_Call: SYSCALL}
			\item \texttt{> sys\_***}
			\item \dots
			\item \texttt{> schedule}
			\item \texttt{> pick\_next\_task <}
			\item \dots
		\end{itemize}
		Il ritorno è del tipo:
		\begin{itemize}
			\item \dots
			\item \texttt{schedule <}
			\item \dots
			\item \texttt{sys\_*** <}
			\item \texttt{System\_Call: SYSRET <}
			\item \texttt{syscall <}
			\item \texttt{*** <}
			\item \texttt{*funzione\_libreria* <}
		\end{itemize}
		L' \texttt{u\_stack} è del tipo:
		\begin{itemize}
			\item \texttt{rientro a *** da syscall}
			\item \texttt{rientro a CU da ***}
			\item \dots
		\end{itemize}
		L' \texttt{s\_stack} è del tipo:
		\begin{itemize}
			\item \texttt{USP}
			\item \dots
			\item \texttt{rientro a System\_Call da sys\_***}
			\item \texttt{PSR U}
			\item \texttt{rientro a syscall da System\_Call}
		\end{itemize}
	\item quando un thread termina l'esecuzione::
		\begin{itemize}
			\item \texttt{fn <}
			\item \texttt{clone} (?) % TODO
			\item \texttt{> syscall}
			\item \dots
		\end{itemize}
	\item le strutture dati HW di un processo contengono:
		\begin{itemize}
			\item \texttt{registro PC}
			\item \texttt{registro SP}: cima dell'ultima pila (sia U o S) 
				utilizzata
			\item \texttt{SSP}: base della pila S
			\item \texttt{USP}: cima della pila U
			\item \texttt{descrittore di stato}: esexuzione, pronto, attesa(di
				cosa?)
		\end{itemize}
\end{itemize}


\newpage
\section*{Memoria virtuale e file system}

\begin{itemize}
	\item \textbf{VMA}: \texttt{area NPV\_iniziale | dim | R/W | P/S | M/A |
		<nome\_file, offset>}
		(il nome del file è \texttt{-1} se la pagina è anonima)
	\item \textbf{PT}: \texttt{process dim : n\_phys\_page -/D R/W}
	\item \textbf{MEMORIA FISICA}: \texttt{n : processo\_area\_n -/D}
	\item \textbf{SWAP FILE}: \texttt{s\_n : processo\_area\_n}
	\item \textbf{TLB}: \texttt{processo\_area\_n : n - D A}
\end{itemize}

\begin{itemize}
	\item \texttt{Dirty bit}
		\begin{itemize}
			\item Finchè non avviene un \texttt{context\_switch} questo viene
				salvato nel TLB.
			\item Appena avviene il context switch il bit viene riportato sulla
				memoria fisica.
			\item Essendo che i file non sono contenuti nel TLB, il bit di dirty
				viene sempre salvato sulla memoria fisica.
			\item nella page table il bit di dirty si mette se e solo se è
				presente anche nella memoria fisica
		\end{itemize}
	\item \texttt{pagine di pila (P)}
		\begin{itemize}
			\item crescono verso il basso, quindi la pagina successiva ha un NPV
				minore di 1.
			\item Nella pila serve sempre avere la prima pagina libera (pagina di
				growsdown).
		\end{itemize}
	\item \texttt{pagine condivise (S)}
		\begin{itemize}
			\item le pagine \texttt{shared} possono essere abilitate alla
				scrittura anche se più processi le condividono.
			\item le pagine non codivise (\texttt{P}), anche se mappate su file,
				quando vengono scritte fanno scattare \texttt{COW}.
		\end{itemize}
	\item \texttt{kswapd}
		\begin{itemize}
			\item gestisce tutte le pagine nelle LRU: la \texttt{active list}
				dalla coda, la \texttt{inactive list} dalla testa
			\item se viene chiamato più volte, i bit di accesso delle pagine
				precedenti rimangono a 1 (come se venissero acceduti anche loro
				ripetutamente)
			\item Il bit di \texttt{A} è salvato nel TLB, quindi pagine non
				presenti nel TLB è come se avessero bit di \texttt{A} a 0.
			\item bisogna quindi porre attenzione allo stato del TLB prima di
				eseguire kswapd: le pagine con bit di A settato saranno in testa
				alla coda, mentre le altre in coda.
			\item Scansione dalla coda della active list:
				\begin{verbatim}
    if (A) {
        A = 0;
        if (ref) move Page to the head of active list;
        else ref = 1;
    } else {
        if (ref) ref = 0;
        else {
	    ref = 1;
	    move Page to the head of inactive list;
	}
    }
				\end{verbatim}
			\item Scansione dalla testa della inactive list:
				\begin{verbatim}
    if (A) {
        A = 0;
        if (ref) {
            ref = 0;
            move Page to the tail of active list;
        } else ref = 1;
    } else {
        if (ref) ref = 0;
        else move Page to the tail of inactive list;
    }
				\end{verbatim}
		\end{itemize}
	\item \texttt{swap file}
		\begin{itemize}
			\item ci vanno solo le pagine anonime e dirty
		\end{itemize}
	\item \texttt{read/write}
		\begin{itemize}
			\item le pagine vengono accedute una alla volta (il \texttt{pfra}
				non sa a prescindere quante pagine saranno accedute).
			\item nella PT \texttt{read} indica che la pagina è stata letta.
			\item nella VMA \texttt{read} indica che la pagina ha permessi di
				sola lettura.
			\item nella PT \texttt{write} indica che la pagina è stata scritta.
			\item nella VMA \texttt{write} indica che la pagina ha permessi di
				lettura e scrittura.
			\item  Le pagine \texttt{W}, dopo una \texttt{fork}, vengono marcate
				come \texttt{R}.
		\end{itemize}
	\item \texttt{swap\_in}
		\begin{itemize}
			\item finchè una pagina non viene scritta rimane duplicata sulla
				swap
			\item se una pagina viene scritta, la pagina del processo che scrive
				subisce la COW, mentre l'altra rimane sia in memoria fisica che
				in swap
			\item se una pagina in swap era "condivisa" tra 2 processi, e uno di
				questi va un \texttt{write}, allora la pagina che viene scritta
				è messa in cima alla \texttt{LRU active} mentre quella non
				scritta in coda alla \texttt{LRU inactive}.
		\end{itemize}
	\item \texttt{sbrk}
		\begin{itemize}
			\item alloca pagine \texttt{D} appena dopo le precedenti, o se non
				erano già presenti appena dopo \texttt{S}.
			\item Essendo le nuove pagina anonime, inizialmente non hanno
				indirizzo, ma si comportano come \texttt{COW} sulla
				\texttt{<ZP>} (una volta lette, sono mappate nella memoria
				fisica sulla \texttt{<ZP>}).
		\end{itemize}
	\item \texttt{mmap(add, n\_pag, W/R, P/S, M/A, file, file\_page\_offset)}
		\begin{itemize}
			\item alloca la pagina specificata, facendo attenzione che i primi
				12 bit di \texttt{address} sono di offset nella pagina (non fanno
				quindi parte della NPV).
		\end{itemize}
	\item \texttt{context switch}
		\begin{itemize}
			\item svouta il TLB: se le pagine sono dirty marca le stesse pagine
				in memoria fisiaca come dirty.
			\item carica prima la pagina del codice e poi quella della pila del
				nuovo processo in memoria fisica (attenzione, sulla pila ci
				scrive anche, quindi nel caso va tolta dalla swap) e quindi
		anche sul TLB. \end{itemize}
	\item \texttt{fork}
		\begin{itemize}
			\item le pagine vengono tutte "condivise" tra i processi
				(\texttt{COW}, \texttt{R}).
			\item tutte le pagine vengono aggiunte in testa alla stessa
				\texttt{LRU} di quelle del padre, con lo stesso bit di ref.
			\item la pagina della pila viene "regalata" al figlio, e clonata
				(\texttt{COW}) dal padre.
			\item la pagina del figlio viene marcata \texttt{D} nella memoria
				fisica (scrive il PID).
			\item il puntatore al descrittore dei file punta alla stessa area di
				memoria.
		\end{itemize}
	\item \texttt{clone}
		\begin{itemize}
			\item crea un nuovo thread: tutte le pagine vengono condivise con
				quella del padre (NON \texttt{COW}).
			\item viene creata una nuova NPV (pila) di tipo \texttt{T}, di
				dimensione 2, W P A. Questa viene marcata \texttt{D} nella
				memoria fisica (come la pila di un nuovo processo in una
				\texttt{fork}).
			\item dopo un \texttt{context\_switch} il TLB viene aggiornato
				(malgrado contenga pagine condivise).
		\end{itemize}
	\item \texttt{pfra}
		\begin{itemize}
			\item prima libera le pagine dei file, e poi quelle dei processi.
			\item quando interviene libera \texttt{MAX\_FREE} pagine.
			\item non sa a prescindere quali pagine verranno usate: potrebbere
				mettere in swap la stessa pagina condivisa per poi ricaricarla
				in memoria fisica.
		\end{itemize}
	\item \texttt{FILE}
		\begin{itemize}
			\item \texttt{READ} indica che la pagina è stata caricata in
				memoria
			\item \texttt{WRITE} indica che la pagina è stata salvata sul disco
				(avviene solo se la pagina è dirty)
			\item se \texttt{f\_count == 0} allora le pagine dirty vengono
				scritte in memoria e la \texttt{f\_pos} si "resetta"
		\end{itemize}
	\item \texttt{fopen}
		\begin{itemize}
			\item se un file viene aperto, allora viene creato un nuovo
				descrittore, e quindi \texttt{f\_pos = 0}.
		\end{itemize}
\end{itemize}

\newpage
\section*{Struttura del filesystem}
\begin{itemize}
	\item prima di aprire un nuovo file, libera una posizione nel
		\texttt{P->files.fd\_array}
\end{itemize}

\newpage
\section*{Tabella della pagine}
\begin{itemize}
	\item l'indirizzo di una pagina è di 48 bit: 9 di offset e 36 di VMA. La
		VMA è divisa in:
		\begin{itemize}
			\item \texttt{PGD}: 9 bit
			\item \texttt{PUD}: 9 bit
			\item \texttt{PMD}: 9 bit
			\item \texttt{PT}: 9 bit
		\end{itemize}
	\item \texttt{\#PGD = 1}
	\item \texttt{\#PUD = \#of\_different\_PGD}
	\item \texttt{\#PMD = \#of\_different\_PUD}
	\item \texttt{\#PT = \#of\_different\_PMD}
	\item \texttt{rapporto\_di\_occupazione = pagine\_totali / numero\_di\_NPV}
	\item \texttt{dimensione\_massima\_del\_processo\_in\_pagine\_virtuali =
		512 * \#PT}
\end{itemize}


\end{document}
