% TODO: riorganizzare
\documentclass[12pt, a4paper]{report}

\usepackage[margin=2cm]{geometry}
\usepackage{ragged2e}

\newtheorem{definition}{Definizione}

\title{Linux}
\author{Gabr1313}
\date{\today}

\begin{document}
\justify
\sloppy

\maketitle
\tableofcontents

\chapter{Introduzione}
\begin{definition}[Posix]
	Standard per i sistemi operativi Unix (usato di seguito sempre).
\end{definition}
\begin{definition}[Processo]
	Macchina virtuale che segue un programma
\end{definition}
\begin{definition}[System service] Particolare funzioni di sistema a cui un
	processo può accedere
\end{definition}
\begin{itemize}
	\item fork, exec, wait, exit
	\item comunicazione tra processi
	\item accesso ai file (periferiche)
\end{itemize}
\section{Thread}
I processi leggeri appartenenti allo stesso processo normale condividono la
stessa memoria, a esclusione dello stack.

Sono identificati dalla coppia \textbf{$<$PID, TGID$>$}, dove PID =
\textit{process ID} e TGID = \textit{Thread group ID}. Le funzioni associate
sono \texttt{getid()} e \texttt{getpid()}.

\section{Sistema operativo}
Il sistema operativo è lo strato che separa l'hardware dall'utente. La usa
funzione principale è permettere l'esecuzione di diversi processi in parallelo
(con 1 sola cpu i processi vengono alternati).
\begin{definition}[Context switch]
	Sostituzione processo in esecuzione
\end{definition}
L'esecuzione di un programma può essere sospesa per 2 motivi:
\begin{itemize}
	\item Sospensione autonoma
	\item Sospensione forzata
\end{itemize}
\begin{definition}[Scheduler]
	Componente del SO che decide quale processo mandare in esecuzione
\end{definition}
\section{Politica di scheduling}
\begin{itemize}
	\item Processi più importanti vengono eseguiti prima
	\item A pari importanza i processi vengono eseguiti in maniera equa
\end{itemize}
Linux è un sistema time sharing
\begin{definition}[Preemption]
	Pausa forzata dal SO di un processo allo scadere di un quanto di tempo
\end{definition}
\begin{definition}[Kernel non-preemptable]
	La preemption è proibita quando un processo esegue servizi del sistema
	operativo
\end{definition}
\begin{definition}[Kernel module]
	Moduli software inseriti nel sistema per permettere l'accesso alle
	periferiche senza la necessità di ricompilare il sistema. Sono inoltre
	caricati dinamicamente durante l'esecuzione quando necessario.
\end{definition}
\section{Architettura}
I file che contengono codice dipendente dall'architettura in \texttt{linux/arch}
D'ora in poi si farà riferimento all'architettura \textbf{x86-64}.
\section{Strutture dati per la gestione dei processi}
\subsection{Descrittore del processo}
La funzione \texttt{get\_current()} restituisce un puntatore al task corrente
\begin{verbatim}
struct task_struct {
    pid_t pid;
    pid_t tgid;
    volatile long state; // ‐1 unrunnable, 0 runnable, >0 stopped
    void *stack; //puntatore all' sStack del task
    ...
}

struct thread_struct {
    unsigned long sp0;
    unsigned long sp; // puntatore allo stack di modo S
    unsigned long usersp; // puntatore allo stack di modo U (idem)
    ...
}
\end{verbatim}
\subsection{sStack (sPila)}
La sPila è diversa per ogni processo

\chapter{Meccanismi Hardware di supporto}

\section{Considerazioni generali}
\begin{enumerate}
	\item PC = program counter
	\item SP = stack pointer
	\item push o pop dalla pila sono un'unica istruzione
	\item chiamate a funzioni salvano in automatico sulla pila il return address
	\item PSR = Process state register: semplificazione di una struttura dati HW
		che contiene le informazioni riguardanti un processo
	\item SSP = Supervisor stack pointer
	\item USP = User stack pointer
\end{enumerate}

\section{Modi di funzionamento}
\begin{itemize}
	\item \textbf{Utente (U)}/Non privilegiato: non può eseguire istruzioni
		privilegiate e può accedere solo alle proprie zone di memoria
	\item \textbf{Supervisore (S)}/Kernel/Privilegiato: può eseguire anche le
		istruzioni privilegiate, a accedere alle zone di memoria di qualsiasi
		processo
\end{itemize}
\section{Memoria virtuale}
Lo spazio di indirizzamento è $2^{64}$ byte. Di questi sono utilizzabili solo i
primi $2^{47}$ byte da U e gli ultimi $2^{47}$ byte da S. Gli altri indirizzi di
memoria sono detti \textit{non-canonical area}.

Una pagina di memoria è di $4Kb$. Ogni indirizzo virtuale prodotto dalla CPU
viene mappato a un indirizzo fisico grazie alla \textbf{Tabella delle Pagine}.

\section{Syscall}
\begin{definition}[Syscall]
	Particolare chiamata a funzione eseguita da un processo quando ha bisogno
	di un servizio dell'OS.
\end{definition}
\begin{definition}[Vettore di Syscall]
	Struttura dati HW costituita da 2 celle, inizializzata dal SO durante
	l'avvio.
\end{definition}
\begin{itemize}
	\item PC: indirizzo della funzione \texttt{system\_call()}
	\item PSR
\end{itemize}
Una sycall quindi, differentemente da una chiamata a funzione, non indica
l'indirizzo a cui deve saltare, ma questo è contenuto nel vettore di syscall.

Anche i valori di SSP e USP sono salvati in una struttura dati HW.

Una Syscall permette di passare dallo stato U allo stato S svolgendo le seguenti
operazioni:
\begin{enumerate}
	\item USP = SP
	\item SP = SSP
	\item push del PC($\to$UPila) sulla SPila
	\item push del PSR sulla SPila
	\item caricamento di PC($\to$SPila) e PSR contenuti nel vettore di syscall
\end{enumerate}

\begin{definition}[Sysret]
	Inverso della syscall.
\end{definition}
Una Sysret, eseguita all fine di una \texttt{system\_call()} permette di passare
dallo stato S allo stato U svolgendo le seguenti operazioni:
\begin{enumerate}
	\item pop del PSR dalla SPila e ripristino registro
	\item pop del PC($\to$UPila) dalla SPila e ripristino registro
	\item SP = USP
\end{enumerate}

\section{Interrupt}
A un insieme di eventi HW è associata un funzione, detta \textbf{Routine di
Interrupt}. Il Meccanismo con cui funziona una Routine di Interrupt è lo
stesso delle syscall, con però alcune differenze:
\begin{itemize}
	\item Sono asincrone: attivate da eventi e non esplicitamente da un
		programma
	\item l'istruzione di ritorno è chiamata IRET
	\item il vettore di syscall è sostituito con la \textbf{Tabella degli
		Interrupt}
\end{itemize}
Se il processo in esecuzione è in modalità S, PC e PSR vengono comunque
salvati sulla sPila.
\begin{definition}[Tabella degli Interrupt]
	Struttura dati HW che contiene i \textbf{vettori di interrupt},
	inizializzata dal SO durante l'avvio.
\end{definition}
\begin{definition}[Vettore di Interrupt]
	Struttura dati HW costituita da 2 celle
\end{definition}
\begin{itemize}
	\item PC: indirizzo della funzione \texttt{system\_call()}
	\item PSR
\end{itemize}
Un meccanismo HW converte l'ID dell'interrupt nell'indirizzo del corrispondente
vettore di interrupt.

La gestione degli errori durante i programmi spesso viene trattata come un
interrupt che termina forzatamente il programma eliminando il processo (abort).

\subsection{Priorità e interrupt nidificati}
Nel PSR, a ogni processo è associato un livello di priorità, e identicamente
funziona per gli interrupt. Un interrupt viene accettato solo se il suo livello
ha priorità superiore rispetto al processo in esecuzione, altrimenti è tenuto
in sospeso. sospeso.

\subsection{ABI}
Gli eseguibili normalmente sono caricati dal SO, mentre gli eseguibili del
sistema stesso devono essere in grado di partire autonomamente (bootstrap).
\begin{definition}[Application Binary Interface]
	Regole che il compilatore utilizza durante la traduzione (indipendentemente
	dall'architettura HW), in modo che tutti i moduli siano coerenti.
\end{definition}
Funzioni della libreria glibc come \texttt{fork()} e \texttt{open()} sono dei
wrapper attorno alla funzione
\begin{verbatim}
	long syscall(long syscall_number, ...);
\end{verbatim}
La chiamata assembly corrispondente è composta da varie istruzioni, l'ultima è
la \texttt{SYSCALL}.

\chapter{Gestione dello stato dei processi}
Ogni processo può trovarsi in uno dei 2 stati:
\begin{itemize}
	\item \textbf{Attesa}: attende un evento
	\item \textbf{Pronto}: può essere eseguito se selezionato dallo scheduler.
		Il processo in esecuzione è chiamata \textbf{processo corrente}.
\end{itemize}
Un processo in esecuzione abbandona tale stato a causa di un evento:
\begin{itemize}
	\item un servizio di sistema richiesto dal processo deve attendere un
		evento. Il processo passa allo stato di attesa.
	\item \textbf{Preemption}: il SO decide di sospendere l'esecuzione a favore
		di un altro processo. Il processo rimane nello stato di pronto.
\end{itemize}
\begin{definition}[Scheduler]
	Componente del SO che esegue le seguenti funzioni:
\end{definition}
\begin{itemize}
	\item Politica di scheduling: quale processo mettere in esecuzione e per
		quanto tempo
	\item Context switch: \texttt{schedule()}
\end{itemize}
Lo scheduler deve quindi gestire la struttura dati \texttt{runqueue}, la quale
contiene 2 campi:
\begin{itemize}
	\item \textbf{RB} (red-black tree): lista dei puntatori ai descrittori dei
		processi pronti
	\item \textbf{CURR}: puntatore al descrittore processo corrente
\end{itemize}
\subsection{SPila}
Linux su x86-64 assegna a ogni processo una sPila di 2 pagine (8Kb). Questa è
utilizzata soltanto durante l'esecuzione di un servizio di sistema.

I valori di SSP e USP sono diversi per ogni processo: prima che avvenga un
context switch vengono salvati nel descrittore del processo.
\begin{itemize}
	\item \textbf{sp0}: indirizzo di base della SPila
	\item \textbf{sp}: valore di SP nel momento in cui il processo viene sospeso
\end{itemize}

\section{Context switch}
Quando P è in modalità U SSP(processo) = sp0(descrittore).\\
Quando P è in modalità S USP contiene il valore di ritorno sulla UPila.

Il context switch può avvenire solo in modalità S (vedi \ref{preemption}),
quindi:
\begin{enumerate}
	\item push di USP sulla sPila
	\item sp(descrittore) = SP(processo)
\end{enumerate}
Ovviamente quando P riprende l'esecuzione si eseguono le operazioni inverse:
\begin{enumerate}
	\item SP(processo) = sp(descrittore)
	\item pop USP dalla sPila e ripristino
	\item SSP(processo) = sp0(descrittore)
\end{enumerate}


\section{Preemption} \label{preemption}
Essendo il Kernel non-preemptive (problemi durante il cambio di contesto mentre
sono in esecuzione routine di interrupt annidate, ad esempio quando un processo
viene iniziato da un processo, interrotto e rieseguito da un altro), se un task
dovrebbe essere sospeso mentre si trova in modalità S, questo non avviene ma la
flag \texttt{TIF\_NEED\_RESCHED} viene messa a 1. Al ritorno in modalità U
avverrà il context switch.

Paradossalmente il SO può decidere di eseguire una preemption solo in modalità S
(aka l'esecuzione può essere abbandonata solo dallo stato S). Si adotta quindi
la seguente regola: se il processo è in modalità S il context switch avviene
alla fine della routine la cui ultima istruzione fa tornare il processo in
modalità U (SYSRET/IRET).


\section{Gestione degli interrupt}
La routine di interrupt è \textbf{trasparente} al processo: l'interrupt è
eseguito nel contesto del processo di esecuzione.
La routine di interrupt può risvegliare dei processi in stato di attesa.

Un interrupt può avvenire in 3 situazioni:
\begin{itemize}
	\item P è in modalità U
	\item P è in modalità S
	\item P sta già eseguendo una routine di interrupt
\end{itemize}
\section{Attesa/Pronto}
Le attese possono essere di 3 tipi:
\begin{itemize}
	\item I/O
	\item sblocco di un lock (mutex e semafori)
	\item scadere di un timer
\end{itemize}

\begin{definition}[Wait\_queue]
	Struttura dati che contiene i descrittore dei processi in attesa di un certo
	evento.
\end{definition}
\begin{definition}[Identificatore dell'evento]
	Indirizzo della wait\_queue.
\end{definition}
La macro \texttt{DECLARE\_WAIT\_QUEUE\_HEAD} dichiara una wait\_queue.

Ad ogni wait\_queue è associata una flag che indica la tipologia di attesa:
\begin{definition}[Attesa esclusiva]
	Un solo processo viene risvegliato all'avvenire dell'evento.
\end{definition}
\begin{definition}[Attesa non esclusiva]
	Tutti i processi vengono risvegliati all'avvenire dell'evento.
\end{definition}
I processi in attesa esclusiva sono inserite alla fine della coda.

Le funzioni associate sono: \texttt{wait\_event\_interruptible\_exclusive()} e
\texttt{wait\_event\_interruptible()}

La funzione \texttt{wake\_up()} risveglia i processi in attesa, mettendoli in
stato di pronto.
\begin{enumerate}
	\item Cambia lo stato da attesa a pronto
	\item Elimina il/i processo/i dalla wait\_queue e li inserisce nella
		runqueue
\end{enumerate}
\section{Segnali}
\begin{definition}[Segnale]
	Avviso asincrono inviato a un processo dal SO o da un altro processo.
\end{definition}
I segnali sono numerati da 1 a 31.

I segnali che non possono essere bloccati sono:
\begin{itemize}
	\item SIGKILL
	\item SIGSTOP
\end{itemize}
Altri segnali sono:
\begin{itemize}
	\item SIGINT: terminazione del processo (CTRL+C)
	\item SIGSTOP: sospensione del processo (CTRL+Z)
\end{itemize}
Quando il segnale viene inviato, il processo può essere in uno dei seguenti
stati:
\begin{enumerate}
	\item \textbf{Esecuzione U}: il segnale viene processato immediatamente
	\item \textbf{Esecuzione S}: il segnale viene processato al ritorno in
		modalità U
	\item \textbf{Pronto}: il segnale viene processato quando il processo
		ritorna in esecuzione
	\item \textbf{Attesa}:
		\begin{itemize}
			\item se l'attesa è \textbf{interrompibile} il processo viene
				risvegliato (pronto). Al risveglio il processo deve
				controllare se la condizione di attesa è diventata falsa, e
				in caso contrario tornare in attesa\\
				\texttt{wait\_interruptible()}\\
				\texttt{wait\_killable()}
			\item se l'attesa è \textbf{non interrompibile} il segnale rimane
				pendente\\
				\texttt{wait\_event()}
		\end{itemize}
\end{enumerate}

\section{Esempio: Driver di periferica}
\begin{enumerate}
	\item P (mod U) richiede un servizio alla periferica PX
		(\texttt{write()})
	\item Syscall (mod S): il SO chiama la funzione del gestore X
		(\texttt{X.write()}) nel contesto di P(write): non avviene il
		context switch
	\item \label{driverloop1} X copia dal buffer dallo spazio del processo
		P al buffer di X C caratteri: \texttt{C = min(buffer.size(),
		char\_to\_be\_written)}
	\item \label{driverloop2} \texttt{X.write()} invia il primo carattere
	\item \texttt{X.write()} chiama \texttt{wait\_event\_interruptible()}
	\item quando PX termina l'operazione genera un interrupt
	\item Routine di interrupt di X: se nel buffer esistono altri caratteri,
		$\to$ \ref{driverloop2}
	\item \texttt{wake\_up()}: Se ci sono altri caratteri da scrivere $\to$
		\ref{driverloop1}
	\item \texttt{X.write()} chiama la Sysret (mod U)
\end{enumerate}

\section{Esempio: Mutex}
Quando un mutex è libero, per evitare di fare una system call (e cioè passare
alla modalità S e mettersi in attesa), Linux utilizza i \textbf{Futex}
(Fast Userspace Mutex). Un futex possiede una \textbf{varibile intera} nello
spazio U e una \textbf{waitqueue} nello spazio S.
\begin{enumerate}
	\item Test e incremento della variabile intera in maniera atomica (in x86-64
		esiste un'istruzione  apposita)
	\item Se la variabile è bloccata:
		\begin{itemize}
			\item \texttt{sys\_futex(FUTEX\_WAIT,...)}
			\item \texttt{wait\_event\_interruptible\_exclusive(WQ, ...)}
		\end{itemize}
	\item Per sbloccare un mutex:
		\begin{itemize}
			\item \texttt{sys\_futex(FUTEX\_WAKE,...)}
			\item \texttt{wakeup(WQ)}
		\end{itemize}
\end{enumerate}
(\texttt{sys\_futex} non non è una funzione, ma un numero sulla tabella delle
routine delle syscall)


\section{Attesa di exit e di un timeout}
Nel caso dei futex e dei driver di periferica la funzione di wakeup conosce la
coda dell'evento. Ci sono tuttavia 2 casi in cui ciò non avviene. In questi
casi viene utilizzata la funzione \texttt{wake\_up\_process()}, la quale riceve
in ingresso il puntatore al descrittore del processo da risvegliare.

I 2 casi sono:
\begin{itemize}
	\item \textbf{Attesa di exit}: processo padre chiama \texttt{wait()}. Il
		figlio chiama \texttt{exit()}. Il figlio nel descrittore possiede il
		puntatore al processo padre da risvegliare.
	\item \textbf{Attesa di un timeout}: Ogni elemento della lista dei timeout
		possiede il puntatore al processo da risvegliare. Una particolare
		funzione (ogni tanto) controlla lo scadere dei timer e risveglia i
		processi.
\end{itemize}

\section{Le funzioni dello scheduler}
\begin{itemize}
	\item \texttt{schedule()}: se in attesa chiama \texttt{deque\_task()}.
		Successivamente esegue il context switch.
	\item \texttt{check\_preempt\_curr()}: se il task deve essere preempted
		\texttt{TIF\_NEED\_RESCHED = 1}.
	\item \texttt{enqueue\_task()}: inserisce il task nella runqueue
	\item \texttt{dequeue\_task()}: rimuove il task dalla runqueue
	\item \texttt{resched()}: \texttt{TIF\_NEED\_RESCHED = 1}
	\item \texttt{task\_tick()}: scheduler periodico invocato dall'interrupt del
		clock. Aggiorna vari contatori e determina se il task debba essere
		preempted.
\end{itemize}

\section{Curiosità}
\begin{itemize}
	\item In sistemi multiprocessore il problema di esecuzioni concorrenti è
		risolto usando meccansimi di sincronizzazione (lock). Il kernel,
		per sempilicità, rimane non preemptive.
	\item In sistemi multiprocessore per attese brevi invece che utilizzaire i
		Mutex (i quali richiedo un context switch) utilia gli
		\textbf{Spinlock}: il task entra in un loop finchè non ottine il lock
		(ha senso solo in sistemi multiprocessore).
	\item Esistono altri stati di un processo: \texttt{TASK\_STOPPED},
		\texttt{EXIT\_ZOMBIE} e \texttt{EXIT\_DEAD}. Questi processi non
		appartenengono a nessuna \texttt{runqueue} o \texttt{waitqueue} e sono
		acceduti tramite \texttt{PID}.
\end{itemize}

\chapter{I servizi di sistema}
\section{Bootstrap}
Le operazioni avviamento successive alla creazione del primo processo, sono
svolte dal programma \texttt{init}, un programma user che crea un processo per
ogni terminale sul quale potrebbe essere eseguito un login. Se  il login va a
buon fine viene lanciato il programma \texttt{shell} (interprete comandi).

\section{Idle}
\texttt{Idle} è il processo 1, che non svolge alcuna funzione utile dopo aver
concluso l'avvio del sistema. Infatti dopo aver concluso le operazioni di
avviamento:
\begin{itemize}
	\item i diritti di esecuzione di \texttt{Idle} sono inferiori a quelli di
		qualsiasi altro processo
	\item non è mai in stato di attesa
\end{itemize}

\section{System Call Interface}
La funzione \texttt{system\_call()} controlla che il numero nel registro
\texttt{rax} sia valido, e lo utilizza come offset su una tabella che contiene
l'indirizzo della routine da eseguire (come una \texttt{switch}).

\section{Creazione dei processi (normali e leggeri)}
La libreria più utilizzata è la \textbf{Native Posix Thread Library}.
I processi noramli sono creati da una \texttt{fork()}, quelli leggeri da una
\texttt{thread\_create()}.

La funzione  \texttt{clone()} permette d creare un processo con caratteristiche
di condivione.

\subsection{sys\_clone}
\texttt{sys\_clone} non è una funzione, ma un numero sulla tabella delle
routine delle syscall.

Il servizio di sistema chiamata da \texttt{clone()} è \texttt{sys\_clone()}
\begin{verbatim}
long sys_clone(unsigned long flags, void *child_stack, ...);
\end{verbatim}
La uPila del figlio è un clone di quella del padre. In particolare se
\texttt{child\_stack = NULL} l'inidirizzo virtuale di USP è lo stesso di quello
del padre, altrimenti è quello specificato.

\subsection{Implementazione di fork()}
\begin{verbatim}
pid_t fork() {
    ...
    syscall(sys_clone, ...); /* no falgs */
    ...
}
\end{verbatim}

\subsection{Implementazione di clone()}
\begin{verbatim}
clone(void (*fn)(void *), void *child_stack, int flags, void *arg, ... ) {
    //push arg e indirizzo di fn utilizzando child_stack
    syscall(sys_clone, ...);
    if (child) {
        //pop arg e fn dalla pila
        fn(arg);
        syscall(sys_exit, ... );
    }
    else return;
}
\end{verbatim}

\section{Eliminazione dei processi}
\begin{itemize}
	\item \texttt{sys\_exit}
	\item \texttt{sys\_exit\_group}: inivia il \texttt{signal} di terminazione
		ai processi del gruppo ed esegue poi la \texttt{sys\_exit}
\end{itemize}
\texttt{sys\_exit} e \texttt{sys\_exit\_group} non sono una funzione, ma un
numero sulla tabella delle routine delle syscall.
\begin{verbatim}
sys_exit(code) {
    struct task_struct *tsk = current( ) // il processo che esegue exit
    exit_mm(tsk);      // rilascia la memoria del processo
    exit_sem(tsk)      // rimuovi il processo dalle code per semafori
                       // o mutex (post su semafori, unlock su mutex)
    exit_files(tsk)    // rilascia i files

    //notifica il codice di uscita al processo padre

    wakeup_up_parent(tsk->p_pptr)  //invoca wake_up del padre
    schedule( );       // esegui un nuovo processo
}
\end{verbatim}
La terminazione del singolo thread termina con \texttt{sys\_exit} (vedi
implementazione \texttt{clone()}.

La funzione \texttt{exit()}, usata per terminare un processo con i suoi thread,
invece, è implementata invocando \texttt{sys\_exit\_group()}, eliminando i
processi con lo stesso TGID.

\chapter{Lo scheduler}
Il comportamento dello scheduler è orientato a garantire le seguenti
condizioni:
\begin{itemize}
	\item Processi più importanti vengono eseguiti prima
	\item A pari importanza i processi vengono eseguiti in maniera equa
\end{itemize}
\begin{definition}[Politica Round Robin]
	Assegnare a ogni processo uno stesso quanto di tempo in ordine circolare.
\end{definition}
\begin{definition}[Priorità]
	Diritto di esecuzione
\end{definition}
I momenti in cui lo scheduler viene invocato sono:
\begin{itemize}
	\item un processo si auto-sospende
	\item un processo viene risvegliato
	\item scadere del quanto di tempo
\end{itemize}

\section{Requisiti dei processi}
\begin{enumerate}
	\item \textbf{Real-time}: garanzia che non vengano eseguiti con un ritardo
		superiore a un certo limite
	\item \textbf{semi-Real-time}: rapidità di risposto senza vincoli
		stringenti sul ritardo
	\item \textbf{normali}:
		\begin{itemize}
			\item \textbf{I/O bound}
			\item \textbf{CPU bound}
		\end{itemize}
\end{enumerate}

\section{Classi di Scheduling}
Diversi processi possono avere diverse politiche di scheduling. Queste sono
realizzate da una \textbf{Scheduler Class}, il cui puntatore è salvato nel
descrittore del processo. Ad esempio la classe \texttt{fair\_sched\_class}:
\begin{verbatim}
static const struct sched_class fair_sched_class = {
    .next               = &idle_sched_class,
    .enqueue_task       = enqueue_task_fair,
    .dequeue_task       = dequeue_task_fair,
    .check_preempt_curr = check_preempt_wakeup,
    .pick_next_task     = pick_next_task_fair,
    .put_prev_task      = put_prev_task_fair,
    .set_curr_task      = set_curr_task_fair,
    .task_tick          = task_tick_fair,
    .task_new           = task_new_fair,
};
\end{verbatim}
Quando ad esepmio un processo in mod S chiama \texttt{enqueue\_task()}, questa a
sua volta chiama \texttt{p->sched\_class->enqueue\_task()}.

Le 3 classi più importanti sono, in ordine di diritto di esecuzione:
\begin{itemize}
	\item \texttt{SCHED\_FIFO}
	\item \texttt{SCHED\_RR}
	\item \texttt{SCHED\_NORMAL}
\end{itemize}
La funzione \texttt{schedule()} invoca \texttt{pick\_next\_task()}, la quale
invoca l'istruzione specifica della singola classe. Poi procede al context
switch.
\begin{verbatim}
schedule() {
	struct tsk_struct *prev = CURR;
	if (prev->stato == ATTESA) dequeue(prev);
	// if (prev terminato) dequeue(prev);
	next = pick_next_task(rq, prev);
	if (next != prev) {
		context_switch(prev, next);
		// CURR->START = NOW;
	}
	TIF_NEED_RESCHED = 0;
}

pick_next_task(rq, prev) {
	for (/* class in ordine di importanza */) {
		next = class->pick_next_task(rq, prev);
		if (next != NULL) return next;
	}
}
\end{verbatim}

\section{Scheduling dei processi soft real-time}
Per supportare i processi RT, Linux utilizza \texttt{SCHED\_FIFO} e
\texttt{SCHED\_RR}. A ogni processo è attribuita una \textbf{priorità statica}
(da 1 a 99), la quale è memorizzata in \texttt{tast\_struct.static\_prio}.
In entrambi i casi processi con maggiore priorità hanno diritto di esecuzione.
Solo nel caso di processi con uguale priorità le politiche differiscono.

\section{Scheduling dei processi normali}
Lo scheduler dei processi normali è chiama \textbf{Completely Fair
Scheduler (CFS)}. IL CFS deve determinare ragionevolmente la durata dei
quanti, assegnare più tempo ai processi più importanti e permettere a processi
in attesa di tornare rapidamente in esecuzione.

Supponendo che tutti i task abbiamo \texttt{t.LOAD = 1} e che questi non si
autosospendano. Il CFS determina un periodo \textbf{PER} all'interno del quale
ognuno degli \textbf{NRT} processi riceve un quanto di tempo \textbf{Q}.
I task sono ordinati in un \textbf{red-black tree}.
\[
	\texttt{PER = max(LT, NRT * GR)}
\]
dove \textbf{LT} è la latenza (default 6ms) e \textbf{GR} è il granularità
(default 0.75ms).

Per calcolare Q, invece, si utilizza una media pesata, dove \texttt{RQL} e' run
queue load e \texttt{LC} e' load coefficient:
\[
	\texttt{RQL = } \sum \texttt{LOAD}
\]
\[
	\texttt{LC = t.LOAD / RQL}
\]
\[
	\texttt{t.Q = PER * t.LC = PER * t.LOAD / RQL}
\]

\subsection{Virtual Time}
\begin{definition}[Virtual Runtime (VRT)]
	Misura del tempo di esecuzione di un processo modificato in base a opportuni
	coefficienti
\end{definition}
Sia \texttt{DELTA} il tempo che un task viene eseguito in $ns$. \texttt{SUM}
rapprensenta il tempo effettivamente trascorso in esecuzione per ogni processo.
\[
	\texttt{t.VRTC = 1 / t.LOAD}
\]
\[
	\texttt{t.VRT = t.VRT + DELTA * t.VRTC = t.VRT + DELTA / t.LOAD}
\]
\[
	\texttt{SUM = SUM + DELTA}
\]
Si ottiene così l'effetto che VRT aumenti più velocemente nei processi più
leggeri (i processi più pesanti utilizzano la CPU per più tempo a ogni ciclo).

Inoltre, la runqueue aggiorna continuamente la variabile \texttt{VMIN} (LFT =
left of RB) (il perche' del \texttt{max(...)} e' spiegato in seguito):
\[
	\texttt{VMIN = max(VMIN, min(CURR.VRT, LFT.VRT))}
\]
La funzione \texttt{tick()} risulta essere quindi
\begin{verbatim}
tick() {
    DELTA = NOW - CURR->START;
    CURR->SUM = CURR->SUM + DELTA; // current time
    CURR->VRT = CURR->VRT + DELTA * CURR.VRTC;
    CURR->START = NOW;
    VMIN = max(VMIN, min(CURR.VRT, LFT.VRT)); // FALSO TODO
    if (CURR->SUM - CURR->PREV >= Q) resched(); // è scaduto il quanto di tempo
}
\end{verbatim}
dopodichè \texttt{schedule()} chiama \texttt{pick\_next\_task()} che eventualmente
chiama \texttt{pick\_next\_task\_fair()}
\begin{verbatim}
pick_next_task_fair(rq, previous_task) {
    if (LFT != NULL) {
        CURR = LFT;
        // elimina LFT da RB ristrutturando la coda opportunamente
        CURR‐>PREV = CURR‐>SUM;
    } else if (CURR == NULL) { // CURR è stato eliminato perchè in ATTESA
        CURR = IDLE;
    }
    return CURR;
}
\end{verbatim}

\section{Gestione di wait e wakeup}
Un processo risvegliato dall'attesa e' plausibile abbia un \texttt{VRT} molto
basso: potrebbe percio' essere modificato. Inoltre modifica i valori di
\texttt{NRT} e \texttt{RQL}, e quindi \texttt{LC} e \texttt{Q} di ogni processo.
\[
	\texttt{tw.VRT = max(tw.VRT, (VMIN - LT/2))}
\]
Dato che e' opportuno che la crescita di \texttt{VMIN} sia monotona:
\[
	\texttt{VMIN = max(VMIN, min(CURR.VRT, LFT.VRT))}
\]

\subsection{Necessita' di rescheduling}
Gli eventi WAIT implicano sempre un rescheduling, mentre gli eventi WAKEUP
eseguono un rescheduling se:
\begin{itemize}
	\item la classe del processo risvegliato ha priorita' maggiore, oppure
	\item \texttt{tw‐>vrt + WGR * tw‐>load\_coeff < CURR‐>vrt} dove
		\texttt{WRG} (wakeup granularity) e' solitamente impostato a 1ms.
\end{itemize}

\subsection{Creazione e cancellazione di task}
\texttt{exit()} e \texttt{clone()} implicano un ricalcolo dei parametri della
runqueue. In particolare durante una clone
\begin{itemize}
	\item \texttt{tnew} viene creato
	\item i paramtetri della runqueue vengono ricalcolati
	\item \texttt{tnew.VRT = VIMN + tnew.Q * tnew.VRTC}: motivo per cui i valori
		di \texttt{VRT} possono differire molto dai valori di \texttt{SUM}
\end{itemize}
Il nuovo processo non viene inserito in prima posizione nel RB.

Il rescheduling viene valutato come se avvenisse una WAKEUP.

\section{Assegnazione dei pesi ai processi}
\begin{definition}[Nice Value]
	Valore assegnabile dall'utente: -20 e' massima priorita', 19 e' minima.
\end{definition}
\[
	\texttt{t.LOAD = 1024 / (1.25 $^{\texttt{t.NICE}}$)}
\]
\end{document}
