\documentclass[12pt, a4paper]{report}

\usepackage{amsmath}
\usepackage{amssymb}
\usepackage{ragged2e}
\usepackage{amsthm}

\newtheorem{definition}{Definizione}
\newtheorem{theorem}{Teorema}
\newtheorem{exercise}{Esercizio}

\title{Appunti}
\author{Gabr1313}
\date{\today}


\begin{document}
\justify

\maketitle
\tableofcontents

\chapter{Matrice Hessiana}
\begin{proof}
	Criterio matrice Hessiana
\end{proof}

\section{Strategie}
Strategie da adottare in caso di forma quadratica indotta da
$H_f(\underline{x}_0)$ sia semidifenita:
\begin{enumerate}
	\item \textbf{Indagine diretta}
	\item Convessità/concavità
\end{enumerate}
\subsection{Indagine diretta}
La startegia consiste nel trovare due curve passanti per il punto critico in cui
le restirzioni di $f$ hanno in un caso un massimo e nell'altro un minimo. Si
conclude quindi che il punto critico è \textbf{punto di sella}.
\subsection{Convessità/ concavità}
\begin{definition}[Funzione concava]
	Sia $f : \mathbb{R}^2 \to R \in \mathcal{C}^2(\mathbb{R}^2)$,
	diciamo che $f$ è covessa (risp. concava) se $\forall (x,y) \in
		\mathbb{R}^2$ la matrice Hessiana $H_f(x,y)$ è definita positiva o
	semi-definita positiva (risp. definita negativa o semi-definita negativa)
\end{definition}
\paragraph{Osservazioni}
\begin{itemize}
	\item Esiste anche la definizione per funzioni non regolari come
	      generalizzazione del caso $n = 1$.
\end{itemize}
\begin{theorem}[]
	Sia $f \in \mathcal{C}^2(\mathbb{R}^2)$ e $(x_0, y_0) \in \mathbb{R}^2$
	un punto critico di $f$, allora:
	\begin{itemize}
		\item se $f$ è convessa su $\mathbb{R}^2$, $(x_0, y_0)$ è punto
		      di \textbf{minimo assoluto}
		\item se $f$ è concava su $\mathbb{R}^2$, $(x_0, y_0)$ è punto
		      di \textbf{massimo assoluto}
	\end{itemize}
\end{theorem}

\section{Riassunto}
\begin{itemize}
	\item Se $A$ è chiuso e limitato in $\mathbb{R}^2$ e $f : \mathbb{R}^2
		      \to \mathbb{R} \in \mathcal{C}^1(\mathbb{R}^2)$,
	      allora i massimi e minimi sono raggiunti da $f$ in $A$ per il teo.
	      di Weierstrass
	\item se A è aperto, allora i punti estermali:
	      \begin{itemize}
		      \item potrebbero non essere raggiunti in $A$
		      \item se f è deriviabile, condizione necessaria è che siano
		            punti critici per il teormea di Fermat
		      \item se $f \in \mathcal{C}^2(A)$, possiamo applicare i criteri
		            della matrice Hessiana per classificare i punti critici, che
		            però non sono conclusivi nel caso $|H_f(\underline{x}_0)| =
			            0$ (dove $\underline{x}_0$ è punto critico)
	      \end{itemize}
	\item se $|H_f(\underline{x}_0)| = 0$ dove $\underline{x}_0$ e\ punto
	      critico, allora possiamo:
	      \begin{itemize}
		      \item strategia diretta per verificare se $\underline{x}_0$ è un
		            punto di sella
		      \item verificare se $f$ è convessa o concava su $\mathbb{R}^2$ per
		            concludere che $\underline{x}_0$ è punto di minimo o massimo
		            assoluto.
	      \end{itemize}
\end{itemize}

\chapter{OTTIMIZZAZIONE VINCOLATA}
\begin{definition}[]
	Sia $A \subseteq \mathbb{R}^2$ aperto e $f, F \in \mathcal{C}^1(A)$. Sia
	$Z$ l'insieme di livello di $0$ di $F$, cioè $z=l_0^F$ o esplicitamente
	$Z := \{ \underline{x} \in A: F(\underline{x}) = 0\}$ che viene chiamato
	vincolo dell'ottimizzazione.

	Dato $\underline{x}_0 = (x_0, y_0) \in Z$ abbiamo che:
	\begin{enumerate}
		\item $\underline{x}_0$ è punto di massimo (risp. minimo) locale o
		      relativo di $f$ vincolato a $Z$ se $\exists \delta > 0$ :
		      $f(x_0, y_0) \geq f(x,y)$ (risp. $f(x_0, y_0) \leq f(x,y)$)
		      $\forall (x,y) \in B_{\delta}(\underline{x}_0 \cap Z$) e
		      $f(\underline{x}_0)$ si dice massimo
		      (risp. minimo) locale o relativo di $f$ vincolato a $Z$
		\item $\underline{x}_0$ è punto di massimo (risp. minimo) assoluto o
		      globale di $f$ vincolato a $Z$ se $\exists \delta > 0$ :
		      $f(x_0, y_0) \geq f(x,y)$ (risp. $f(x_0, y_0) \leq f(x,y)$)
		      $\forall (x,y) \in B_{\delta}(\underline{x}_0 \cap Z$) e
		      $f(\underline{x}_0)$ si dice massimo (risp. minimo) assoluto o
		      globale di $f$ vincolato a $Z$.
		\item $\underline{x}_0$ è punto estremale o di estremo vincolato a $Z$
		      se i punti di massimo o minimo locale vincolato a $Z$
	\end{enumerate}
\end{definition}

Si parla di ottimizzazione vincolata di $f$ con vincolo $Z$ per indicare la ricerca dei punti estremali di $f$ vincolati a $Z$.





\end{document}
